\section{Trajectory Planning}

\STANDARD{\insertsection}
{ 
  \framesubtitle{\insertsubsection}

  \begin{figure}[!h]
    \centering
    \input{images/s0-nach-s1}
  \end{figure}

  Wir müssen hier unterscheiden zwischen:

  \begin{itemize}
    \item der Beschreibung der Position der Aktoren
    \item und der Beschreibung der Lage des Effektors (Werkzeugs)
      \begin{itemize}
        \item diese wird auch als Pose bezeichnet und kann durch 3 Positionsangaben (wie x,y,z) und 3 Drehwinkel (wie a,b,c) bezogen auf ein Bezugskoordinatensystem beschrieben werden
%Weber, Industrieroboter Seite 16 
        \item sie beschreibt eine Bahn im Raum
      \end{itemize}
  \end{itemize}

  \textbf{Einschränkung}: zunächst nur die Position der Aktoren
}


\STANDARD{\insertsection}
{ 
  \framesubtitle{\insertsubsection}

  \textbf{Aufgabe}: Beziehung zwischen Zeit und Position finden

  \textbf{Synonyme}: Path Planning, Motion Planning

  Unterscheidung hier:

  \begin{itemize}
    \item Geometrie (Path): Position der Aktoren ohne Zeitinformation
    \item Trajektorie (Trajectory): Position, Geschwindigkeit, Beschleunigung und Ruck als Funktion über die Zeit
  \end{itemize}
% Seite 2 Trajectory generation for a four axis robot with linear kinematics - S.N. van den Brink - DCT 2005.69
%With a trajectory we indicate the position, velocity, acceleration and jerk profiles of every actuator as a function of time.
%With a path, the subsequent positions of every actuator without time information is indicated

\textbf{Vereinfachung}

  \begin{itemize}
    \item Eindimensionale Trajektorie: $q=q(t)$
          \\Definiert durch eine Skalar-Funktion
    \item Mehrdimensionale Trajektorie: $\textbf{p}=\textbf{p}(t)$
          \\Definiert durch eine Vektor-Funktion
  \end{itemize}

  \textbf{Einschränkung}: zunächst nur eindimensionale Trajektorien
}

\STANDARD{\insertsection}
{ 
  \framesubtitle{\insertsubsection}

  $q_0$: Startposition

  $q_1$: Zielposition

  \begin{figure}[!h]
    \centering
    \input{images/s0-nach-s1-1dim}
  \end{figure}
}

\section{Notation}
\STANDARD{\insertsection}
{ 
  \framesubtitle{\insertsubsection}

  Position

  \begin{equation}
    q(t)
  \end{equation}

  Geschwindigkeit (Velocity)

  \begin{equation}
    v(t) 
    = \dot{q}(t) 
    = \frac{d}{dt} q(t)
  \end{equation}

  Beschleunigung (Acceleration)

  \begin{eqnarray}
    a(t) 
    = \dot{v}(t) = \frac{d}{dt} v(t) 
    = \ddot{q}(t) = \frac{d^2}{dt^2} q(t)
  \end{eqnarray}

  Ruck (Jerk)

  \begin{eqnarray}
    j(t) 
    = \dot{a}(t) = \frac{d}{dt} a(t)
    = \ddot{v}(t) = \frac{d^2}{dt^2} v(t)
    = q^{(3)}(t) = \frac{d^3}{dt^3} q(t)
    %= \dddot{q}(t) = \frac{d^3}{dt^3} q(t)
  \end{eqnarray}
}


\section{Bang-Bang-Control}

\STANDARD{\insertsection}
{ 
  \framesubtitle{\insertsubsection}

  \textbf{Prozess}: Positionierung

  \textbf{Aufgabe}: Positionieren in möglichst kurzer Zeit

  \textbf{Ansatz}: Höchstmögliches ausreizen der limitierende(n) Größe(n)

  \textbf{Grenzen}: Limitierende Größen (Constraints) ergeben sich

  \begin{itemize}
    \item durch den Motor
          \\über die Höchstdrehzahl wird $v_{max}$ festgelegt
          \\über das Drehmoment wird $a_{max}$ festgelegt
    \item durch die Dynamik des mechanischen Systems
          \\über Steifigkeit/Nachgiebigkeit wird $j_{max}$ festgelegt
    \item durch die Geometrie wird festgelegt, ob
          \\$j_{max}$, $a_{max}$ und $v_{max}$ überhaupt erreicht werden können
      \begin{itemize}
        \item da \textbf{hier} nur eindimensionale Trajektorien betrachtet werden, ist nur die \textbf{Weglänge} der begrenzende Faktor
        \item bei mehrdimensionalen Trajektorien ist die Krümmung ein weiterer begrenzender Faktor
      \end{itemize}
  \end{itemize}
}