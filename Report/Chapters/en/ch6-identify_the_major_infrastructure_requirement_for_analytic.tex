\chapter{Identify the major infrastructure requirement for Analytics.}

\section{Structural requirements.}

\subsubsection{Environment.}
The operation's primary goal is to record a participant's gait. Our study does not need to consider participant-impacting factors like temperature, wind, sound, etc. Given this and our desire to minimize process complexity, a Closed Environment is the best for our study.

\subsubsection{Sensing device.}
Gait identification using wearable motion sensors is a popular issue due to the widespread use of movement sensors. Most wearable motion sensors use \ac{mems} inertial sensors. These MEMS inertial sensors (accelerometers) are coupled to form IMUs. Electromechanical accelerometers measure acceleration forces along one, two, or three axes. IMUs are used for sophisticated motion analysis because they are mobile, compact, and powerful. We considered gait identification using wearable IMU because it's more effective.

\subsubsection{Power source}
The system's power supply is one of our study's key criteria. In our context, a wireless power supply to the research participant is required to keep the procedure simple. Batteries were the greatest choice as a result. Additionally, because the battery would be linked to the body, its size and weight might have an impact on how the person moves. Therefore, taking everything into account, a Li-Po battery with a 3.7V is best in our situation.

\subsubsection{Processing device.}
The central component of our structural need is the processing unit, or "The Brain," of our study. All types of tasks, including storing  the acquired data, filtering the data, processing the data, and making decisions based on the processed data, will be carried out in this unit. Therefore, a computer with 8 Gb RAM, at least 500 Gb of storage, a monitor or display, and an 802.11ac 2.4/5 GHz wireless network adapter would be sufficient as a processing device.

\section{Technological requirements.}
As an ambulatory monitoring solution to handle the gait analysis, accelerometer-based devices have been suggested. Accelerometer-based methods to extract pertinent gait events and gait phases have been suggested in this situation.

\begin{itemize}
\item Basic technological requirements for data transmission include the transmitter and receiver modules that are built inside the Arduino Portenta H7 IMU in addition to the accelerometer..

\item Software requirements are just as important to technological needs as hardware requirements are. Programming languages that can held TensorFlow libraries like Python, as well as other applications like the Arduino IDE and tools for graphical representation of the data that is provided.

\end{itemize}

\section{Behavioural requirements.}
\subsubsection{Scenario.}
The subject must be made system-ready as the first stage in the study. Two Arduino Portenta H7 IMU and a Li-po battery are firmly attached to each foot, or the right foot and left foot, one at the heel level and one at the forefoot. In order to prevent obstructing the subject's motions, the wires connecting the accelerometers and transmitter module (located at the waist level) were firmly secured around the legs. Once the subject is prepared, a walking surface—in this case, a treadmill—is set up in the room.
\bigskip

After the system has been established, the following phase is implementation. On the treadmill, the participant or subject is outfitted with hardware components for walking. In this context, steps per minute represent the walking speed. As soon as the subject begins to walk, the transmitter will transmit the data generated by the Arduino Portenta H7 IMU's sensors. Afterwards, the receiver will receive these data signals (Arduino on board). These acquired data are then stored in the processing unit with the aid of Arduino.
\bigskip

This data is processed using a variety of algorithms as we get closer and closer to the finish line, which ultimately helps in the decision-making process. The knowledge discovery procedure of the database serves as the foundation for the entire data processing system (KDD). Therefore, the entire analytics process begins with selecting the appropriate data, storing it, and finally preparing it. After the data have been adequately structured, they are sent on to the processing part of the \ac{kdd}, which includes data transformation and data mining before the modelling and validation portions of the process. When sufficient information has been gathered, the user will be instructed to get off the treadmill and cut their connection to the foot sensors.
\bigskip